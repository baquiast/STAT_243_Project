\documentclass{article}
\usepackage[left=3cm,right=3cm,top=2cm,bottom=2cm]{geometry} 
\usepackage{amsmath}
\usepackage{graphicx}
\usepackage{float}
\begin{document}
\section*{Evolution of the ARS}
Bounds of the standard normal distribution as made by the ARS are shown in figure \ref{arsplot}. Occasionally as the 
sample size $n$ increases, a new value of the density is computed. Each new point provides a tighter bound of the distribution.
With tighter bounds it is less likely that approving the next sample element requires calculating a distribution value. 

Another thing to note is that the bounds will lie closer to the distribution for values with high density since more values are
chosen from dense parts.  


\begin{figure}[H]
\centering
\includegraphics[width=\textwidth]{normalArs}
\caption{Development of ARS bounds by increasing sample size $n$. A value for the distribution is computed every time a 
candidate for the sample is rejected from the squeezing step. As more density values are known the bounds will move
closer to the distribution.}
\label{arsplot}
\end{figure}
\end{document}
